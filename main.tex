\documentclass{article}
\usepackage{graphicx} % Required for inserting images
\usepackage{ctex}
\usepackage{algorithm}  
\usepackage{algorithmicx}  
\usepackage{algpseudocode}  
\usepackage{amsmath} 
\usepackage{amsthm}
\usepackage{amsfonts}
\usepackage{mathtools}
\usepackage{geometry}
\usepackage{chngcntr}
\usepackage{tikz}
\usepackage{subcaption}
\usepackage{enumitem}
\usepackage{hyperref}
\usepackage[backend=biber,style=gb7714-2015,gbpub=false,gbnamefmt=lowercase]{biblatex}

\addbibresource{main.bib}

\setlist[itemize]{
  itemsep=0pt,
  parsep=0pt,
  topsep=0pt,
  partopsep=0pt
}

\title{算法总结}
\author{Jiayang Sun}
\date{January 2026}

\begin{document}

\maketitle

\section{算法复杂度分析——渐近表示法}

\paragraph{大$\Theta$记号} 对于一个给定函数$g(n)$,$\Theta(g(n))$表示一个函数的集合:
$$
\Theta(g(n))=\{f(n)\mid\exists c_1, c_2, n_0>0, \text{s.t.}\forall n\geq n_0, 0\leq c_1g(n)\leq f(n)\leq c_2g(n)\}
$$

一般记$f(n)\in\Theta(g(n))$为$f(n)=\Theta(g(n))$。下面的记号同理。

\paragraph{大$O$记号} 对于一个给定函数$g(n)$,$O(g(n))$表示一个函数的集合:
$$
O(g(n))=\{f(n)\mid \exists c, n_0>0, \text{s.t.}\forall n\geq n_0, 0\leq f(n)\leq cg(n)\}
$$

\paragraph{大$\Omega$记号} 对于一个给定函数$g(n)$,$\Omega(g(n))$表示一个函数的集合:
$$
\Omega(g(n))=\{f(n)\mid \exists c, n_0>0, \text{s.t.}\forall n\geq n_0, 0\leq cg(n)\leq f(n)\}
$$

\paragraph{小$o$记号} 对于给定函数$g(n)$,$o(g(n))$表示一个函数的集合:
$$
o(g(n))=\{f(n)\mid \forall c>0, \exists n_0>0, \text{s.t.}\forall n\geq n_0, 0\leq f(n)< cg(n)\}
$$

\paragraph{小$\omega$记号} 对于一个给定函数$g(n)$,$\omega(g(n))$表示一个函数的集合:
$$
\omega(g(n))=\{f(n)\mid \forall c>0,\exists n_0>0, \text{s.t.}\forall n\geq n_0, 0\leq cg(n)\leq f(n)\}
$$

\vspace{1em}
证明时只需要设定$c$和$n_0$,然后找出对应的值即可。

\section{分治法}

\subsection{解题思路}

\begin{itemize}
    \item 如何分解子问题(Divide)
    \item 如何递归地解决子问题(Conquer)
    \item 如何合并子问题的解(Combine)
\end{itemize}

\subsection{复杂度分析}

\paragraph{递归树法}

根据递归式画出递归树,计算树的所有节点数即为算法的时间复杂度。

\paragraph{主定理}

对于形如$T(n)=aT(n/b)+f(n)$的递归式,其中$a\geq1, b>1, f(n)$渐近正,记$g(n)=n^{\log_ba}$,其时间复杂度满足:
\begin{itemize}
    \item 若$g(n)$较大($f(n)=O(g(n))$),则$T(n)=\Theta(g(n))=\Theta(n^{\log_ba})$
    \item 若$f(n)$较大($f(n)=\Omega(g(n))$),则$T(n)=\Theta(f(n))$
    \item 若$f(n)$和$g(n)$相当($f(n)=\Theta(g(n))$),则$T(n)=\Theta(g(n)\log n)=\Theta(n^{\log_ba}\log n)$
\end{itemize}

\vspace{1em}
注:不是对所有的递归式均满足。例如以下递归式:
$$
T(n)=2T(n/2)+n\log n
$$
其中$n^{\log_ba}=n<f(n)=n\log n$,但$f(n)$并不大于$n$一个多项式因子$n^\epsilon, \epsilon>0$。

对于给定$\epsilon>0$,对于足够大的$n$,$n^\epsilon >\log n$,因此不能用主定理求解时间复杂度。

\subsection{例题}

\paragraph{斐波那契数列}

$$
F(n)=\begin{cases}
    1\quad n=0 \\
    1\quad n=1 \\
    F(n-1) + F(n - 2)\quad n>1
\end{cases}
$$

\paragraph{全排列问题}

设计一个递归算法生成$n$个元素$\{r_1, r_2, ..., r_n\}$的全排列。

设$R=\{r_1, r_2, ..., r_n\}$为要进行排列的$n$个元素,$R_i=R-\{r_i\}$。
集合$X$中的元素的全排列记为$\text{perm}(X)$,$\text{perm}(X)(r_i)$表示在全排列$\text{perm}(X)$的每一个排列后加上后缀$r_i$得到的排列。则$R$的全排列可以归纳定义为:
\begin{itemize}
    \item 当$n=1$时,$\text{perm}(R)=(r)$,其中$r$是集合$R$中唯一一个元素
    \item 当$n>1$时,$\displaystyle\text{perm}(R)=\bigcup_{i=1}^n\text{perm}(R_i)(r_i)$
\end{itemize}
时间复杂度为$O(n!)$。

\paragraph{整数划分}
给定一个正整数$n$,$n=n_1+n_2+\cdots+n_k$表示正整数$n$的一个$k$划分,其中$n_1\geq n_2\geq \cdots\geq n_k\geq 1$。求正整数$n$的不同划分的个数。

记最大加数$n_1\leq m$的划分的个数为$q(n, m)$,有
\begin{itemize}
    \item 当最大加数$n_1$不大于1时,任何正整数只有一种划分方式:$n=1+1+\cdots 1$,即$q(n, 1)=1, n\geq 1$
    \item 最大加数$n_1$不能大于$n$,即$q(n, m)=q(n, n), m\geq n$
    \item 正整数$n$的划分由$n_1=n$的划分和$n_1\leq n-1$的划分组成,即$q(n, n)=1+q(n, n-1)$
    \item 正整数$n$的最大加数$n_1$不大于$m$的划分由$n_1=m$的划分和$n_1\leq m-1$的划分组成,即$q(n, m)=q(n, m-1)+q(n-m, m), n>m>1$,其中$q(n-m, m)$表示先拿出一个$m$作为分划的一部分,然后再考虑剩下的最大分划数不超过$m$的分划个数$q(n-m, m)$
\end{itemize}
因此有递归关系:
$$
q(n, m)=\begin{cases}
    1\quad &n=1, m=1\\
    q(n, n)\quad &n<m\\
    1+q(n, n-1)\quad &n=m\\
    q(n, m-1)+q(n-m, m)\quad &n>m>1
\end{cases}
$$

\paragraph{$n$阶汉诺塔问题}
\begin{figure}[!h]
    \centering
    \includegraphics[width=0.5\linewidth]{figures/1e540d282acce33bfb8f2a33b752516b.png}
    \caption{$n$阶汉诺塔示意图}
    \label{fig:n-hanoi}
\end{figure}

将$X$上的$n$个圆盘移动到$Z$上需要多少步?

分解问题:对于第$n$个盘:
\begin{itemize}
    \item 将上面$n-1$个盘先移动到$Y$上,以$Z$作为辅助盘(若干步)
    \item 将第$n$个盘移动到$Z$上(一步)
    \item 将上面的$n-1$个盘移动到$Z$上,以$X$为辅助盘(若干步)
\end{itemize}
当$n=1$时,直接将该盘子从$X$移动到$Z$(一步)。

\begin{figure}[!h]
    \centering
    \href{https://upload.wikimedia.org/wikipedia/commons/2/20/Tower_of_Hanoi_recursion_SMIL.svg}{\includegraphics[width=0.5\linewidth]{figures/500px-Tower_of_Hanoi_recursion_SMIL.svg.png}}
    \caption{$n=6$时的解法步骤}
    \label{fig:6-hanoi}
\end{figure}

\textit{思考:对于$4$个柱子应该怎么做?对于$m$个柱子($m<n$)呢?}\cite{menon2025optimalgeneralsolutionmultipeg}

\paragraph{排序问题} 归并排序和快速排序均是基于分治法的排序方法。

\textbf{1. 归并排序}

\textbf{2. 快速排序}

\paragraph{二分搜索} 适用于有序表

\paragraph{快速傅里叶变换(FFT)} 适用于多项式乘法、大整数乘法(整数可以用一个多项式表示)问题。两个$n-1$次的多项式的乘积有$2n-1$个系数。

\textbf{多项式的点值表示法:}记$n-1$次多项式$A(x)=\sum_{j=0}^{n-1}a_jx^j$。则$A(x)$可表示为$n$个点值对$\{(x_k, y_k)\}_{k=0}^{n-1}$,其中$y_k=A(x_k)$。

对应有如下方程组:
\begin{equation}
    \begin{pmatrix}
        1 & x_0 & x_0^2 & \cdots & x_0^{n-1} \\
        1 & x_1 & x_1^2 & \cdots & x_1^{n-1} \\
        \vdots & \vdots & \vdots & \ddots & \vdots \\
        1 & x_{n-1} & x_{n-1}^2 & \cdots & x_{n-1}^{n-1}
    \end{pmatrix}
    \begin{pmatrix}
        a_0 \\
        a_1 \\
        \vdots \\
        a_{n-1}
    \end{pmatrix}
    =\begin{pmatrix}
        y_0 \\
        y_1 \\
        \vdots \\
        y_{n-1}
    \end{pmatrix}
\label{eq:poly-points}
\end{equation}
其中左边为范德蒙矩阵~\cite{wiki:Vandermonde},可记为$V(x_0, x_1, \cdots, x_{n-1})$。

点值表示法的优点:记$A(x)$由$\{(x_k, y_k^A)\}_{k=0}^{n-1}$表示,$B(x)$由$\{(x_k, y_k^B)\}_{k=0}^{n-1}$表示,则$A(x)\times B(x)$在$\{x_0, x_1, ..., x_{n-1}\}$处的值为$\{y_0^Ay_0^B, y_1^Ay_1^B, ..., y_{n-1}^Ay_{n-1}^B\}$。

\textbf{FFT的思路:}对于两个多项式$A(x)$和$B(x)$,首先将两个多项式转换为点值表示,然后将对应的点相乘,最后使用逆变换将相乘后的点值恢复成多项式。

\begin{equation}
    \begin{matrix}
        A(x), B(x) & \xRightarrow{A(x)\times B(x): O(n^2)} & C(x) \\
        \Downarrow\tiny\text{转化} & & \Uparrow\tiny\text{恢复} \\
        \{(x_k, A(x_k))\}_{k=0}^{2n-1}, \{(x_k, B(x_k))\}_{k=0}^{2n-1} & \xRightarrow{\text{点值对相乘}: O(n)} & \{(x_k, C(x_k))\}_{k=0}^{2n-1}
    \end{matrix}
\end{equation}

目标:找到一组合适的点对$\{(x_k, y_k)\}_{k=0}^{n-1}$使得方程组(\ref{eq:poly-points})有且仅有唯一解。$\Rightarrow$找到一组相异的$x_k$:$x_k$相异$\iff V(x_0, x_1, \cdots, x_{n-1})$可逆。

\textit{目的:可以根据点对恢复出原本的多项式。}

\textbf{考虑如何将多项式高效转换为点值表示:}

观察一个$n-1$次多项式
$$
A(x)=a_0x^0+a_1x^1+a_2x^2+a_3x^3+\cdots+a_{n-1}x^{n-1}
$$

Naïve的做法:直接将$n$个$x$带入求解。计算一次$A(x)$需要$O(n)$时间,因此总共的时间复杂度为$O(n^2)$。

\vspace{1em}
原式可以分为奇数项系数和偶数项系数:
\begin{align*}
A(x)&=(a_0x^0+a_2x^2+\cdots+a_{n-2}x^{n-2})+(a_1x^1+a_3x^3+\cdots+a_{n-1}x^{n-1}) \\
 &=(a_0x^0+a_2x^2+\cdots+a_{n-2}x^{n-2})+x(a_1x^0+a_3x^2+\cdots+a_{n-1}x^{n-2}) \\
 &=(a_0(x^2)^0+a_2(x^2)^1+\cdots+a_{n-2}(x^2)^{n/2-1})+x(a_1(x^2)^0+a_3(x^2)^1+\cdots+a_{n-1}(x^2)^{n/2-1})
\end{align*}
记$A_{even}(x)=a_0+a_2x+a_4x^2+\cdots a_{n-2}x^{n/2-1}$,$A_{odd}(x)=a_1+a_3x+a_5x^2+\cdots a_{n-2}x^{n/2-1}$,则有
\begin{equation}
A(x)=A_{even}(x^2)+xA_{odd}(x^2)
\label{eq:ffn-positive}
\end{equation}

可以使用递归的方式先解出$A_{even}(x^2)$和$A_{odd}(x^2)$然后再合并得到$A(x)$。要得到$n$个点对,可以取任意$n$个不同的$x$带入,这样有递归式$T(n)=2T(n)+O(n)$,根据主定理,复杂度仍为$O(n^2)$。有没有什么方法能取$n/2$个不同的$x$带入得到$n$个点对?

观察式(\ref{eq:ffn-positive})可以得到对应$-x$的公式:
\begin{equation}
    A(-x)=A_{even}(x^2)-xA_{odd}(x^2)
\end{equation}

% TODO: finish the discription
\boxed{\text{to finish the discription}}

\paragraph{Strassen矩阵乘法} 对于两个矩阵$A$和$B$,求二者的乘积$C=AB$。

直接计算:
$$
c[i, j]=\sum_{k=1}^nA[i, k]B[k, j]
$$
时间复杂度为$O(n^3)$。

考虑对矩阵进行分块:
$$
\begin{pmatrix}
    C_{11} & C_{12} \\
    C_{21} & C_{22} 
\end{pmatrix}
=
\begin{pmatrix}
    A_{11} & A_{12} \\
    A_{21} & A_{22}
\end{pmatrix}
\begin{pmatrix}
    B_{11} & B_{12} \\ 
    B_{21} & B_{22}
\end{pmatrix}
$$
于是有:
\begin{align*}
    C_{11}=A_{11}B_{11}+A_{12}B_{21} \\
    C_{12}=A_{11}B_{12}+A_{12}B_{22} \\
    C_{21}=A_{21}B_{11}+A_{22}B_{21} \\
    C_{22}=A_{21}B_{12}+A_{22}B_{22}
\end{align*}
将原$n\times n$矩阵乘法转换为8个$n/2\times n/2$矩阵的乘法,于是有递归式:
$$
T(n)=\begin{cases}
    O(1)& n=2 \\
    8T(n/2)+O(n^2) &n>2
\end{cases}
$$
根据主定理,时间复杂度仍为$O(n^3)$。

通过一些手段将8个乘法变成7个乘法:记
\begin{align*}
    M_1=A_{11}(B_{12}-B_{22}) \\
    M_2=(A_{11}+A_{12})B_{22} \\
    M_3=(A_{21}+A_{22})B_{11} \\ 
    M_4=A_{22}(B_{21}-B_{11}) \\
    M_5=(A_{11}+A_{22})(B_{11}+B_{22}) \\ 
    M_6=(A_{12}-A_{22})(B_{21}+B_{22}) \\
    M_7=(A_{11}-A_{21})(B_{11}+B_{12})
\end{align*}
于是有
\begin{align*}
    C_{11}&=M_5+M_4-M_2+M_6 \\
    C_{12}&=M_1+M_2 \\
    C_{21}&=M_3+M_4 \\
    C_{22}&=M_5+M_1-M_3-M_7
\end{align*}
递归式变为
$$
T(n)=\begin{cases}
    O(1)& n=2 \\
    7T(n/2)+O(n^2) &n>2
\end{cases}
$$
时间复杂度为$O(n^{\log7})=O(n^{2.81})$。

\paragraph{平面最近点对} 给定一个二维平面上的$n$个点,求找出其中距离最小的一对点。

易得暴力方法的时间复杂度:$O(n^2)$。

分治的思路:将平面分为两个子平面,递归找出两个子平面中距离最短的点对,记距离为$d_1$和$d_2$,和跨两个子平面的点对的距离对比,得到距离最小的点对。

记$\delta=\min\{d_1, d_2\}$,那么只需要分析分割线左右$\delta$范围即可。

\begin{figure}[!h]
    \centering
    \includegraphics[width=0.5\linewidth]{figures/nearest_point.png}
    \caption{如何找到跨分割线的最近点对}
    \label{fig:nearest-point}
\end{figure}

时间复杂度分析:有递归式$T(n)=2T(n/2)+O(n)$,根据主定理有时间复杂度为$O(n\log n)$。

\paragraph{棋盘覆盖问题} 在一个$2^k\times 2^k$个方格组成的棋盘中,恰有一个方格与其它方格不同,称该方格为一特殊方格,且称该棋盘为一特殊棋盘(如图~\ref{fig:checkboard})。在棋盘覆盖问题中,要用图示的4种不同形态的L型骨牌覆盖给定的特殊棋盘上除特殊方格以外的所有方格,且任何2个L型骨牌不得重叠覆盖。

\begin{figure}[!h]
    \begin{minipage}{0.3\linewidth}
        \centering
        \includegraphics[width=0.8\linewidth]{figures/checkboard.jpg}
    \end{minipage}
    \begin{minipage}{0.7\linewidth}
        \centering
        \includegraphics[width=\linewidth]{figures/domino.jpg}
    \end{minipage}
    \caption{棋盘问题示例}
    \label{fig:checkboard}
\end{figure}

可以将$2^k\times 2^k$的棋盘分割为4个$2^{k-1}\times 2^{k-1}$的子棋盘,必有一个子棋盘为特殊棋盘。可以使用一个骨牌将剩下三个子棋盘转化为特殊棋盘,于是将原本问题转化为4个子问题,如图~\ref{fig:checkboard-solution}所示。可得递归式
$$
T(k)=4T(k-1)+O(1)
$$
时间复杂度为$O(4^k)$。

\begin{figure}[!h]
    \centering
    \includegraphics[width=0.5\linewidth]{figures/checkboard_solution.jpg}
    \caption{递归求解棋盘问题}
    \label{fig:checkboard-solution}
\end{figure}

\paragraph{循环赛日程表问题} 设计一个满足以下要求的日程表:
\begin{itemize}
    \item 每个选手必须与其他$n-1$个选手各赛一次
    \item 每个选手一天只能赛一次
    \item 循环赛一共进行$n-1$天
\end{itemize}

可将所有选手分为两半,$n$个选手的比赛日程表可以通过为$n/2$个选手设计的比赛日程表来决定。递归地分割直到只有两个选手,只需要让这两个选手进行比赛即可。

\section{动态规划}

\subsection{解题思路}

\begin{itemize}
    \item 定义状态
    \item 构造状态转移(递归关系)
    \item 自底向上计算
\end{itemize}

动态规划和分治法的区别:更适合解决有重叠子问题的情况。

\subsection{例题}

\paragraph{最长公共子序列(LCS)}

给定两个长度分别为$m$和$n$的序列$X$和$Y$,若$Z$既是$X$的子序列,又是$Y$的子序列,则$Z$为$X$和$Y$的公共子序列。要求$X$和$Y$的公共子序列中长度最大者。

记$C[i, j]$为$X_i$和$Y_j$的LCS长度。则对于$0\leq i\leq m, 0\leq j\leq n$,有
$$
C[i, j]=
\begin{cases}
    0 & i=0~\text{or}~j=0 \\
    C[i-1, j-1] + 1 & i, j>0, x_i=y_j \\
    \max\{C[i, j-1], C[i-1, j]\} & i, j>0, x_i\neq y_j
\end{cases}
$$
初始化行和列,然后依次由$i=1, 2, ..., m$和$j=1, 2, ..., n$递推计算,最终$C[m, n]$即为所求解。

可通过将$C$压缩为$2\times n$的数组求解,将空间复杂度缩小为$O(m+n)$\footnote{$O(n)$?}。

\paragraph{0-1背包问题}

给定$n$种物品,其中第$i$种物品的价值为$v_i$,重量为$w_i$。现有一背包,可以装总质量不超过$C$的若干物品,问如何选择装入背包的物品,使得在不超出背包容量的情况下,装入背包中的物品总价值最大?

记$m[i, j]$表示背包容量为$j$,可选择物品为$0, 1, ..., i$时的最优解。则有递归式:
\begin{equation}
m[i, j]=\begin{cases}
    \max\{m[i-1, j], m[i-1, j-w_i]+v_i\} & j\geq w_i \\
    m[i-1, j] & 0\leq j<w_i
\end{cases}
\label{eq:bag}
\end{equation}
含义为:若当前背包可以装下第$i$个物品($j\geq w_i$),则比较装物品$i$后的总价值($m[i+1, j-w_i]+v_i$)和不装物品$i$后的总价值($m[i+1, j]$);否则不装物品$i$(总价值和$m[i+1, j]$的总价值相同)。

也可以使用一维动态规划求解:记$m[j]$表示背包容量为$j$时的最大总价值。那么对于每一个物品$i$,有递归式:
$$
m[j]=\begin{cases}
    \max\{m[j], m[j-w_i]+v_i\} & j\geq w_i \\
    m[j] & 0\leq j< w_i
\end{cases}
$$
含义与式(\ref{eq:bag})相似。注意:这里的$j$需要从$C$到1遍历。

\section{贪心算法}

\subsection{例题}

\paragraph{活动选择问题}

设有$n$个活动$S={a_1,a_2,...,a_n}$,均要使用某资源(如同一间教室),该资源使用方式为独占式,一次只供一个活动使用。每个活动$a_i$发生的时间为$[s_i,f_i),0\leq s_i<f_i<\infty$。两个活动相容(不冲突),是指其中一个活动的开始时间必须大于等于另一个活动的完成时间。问如何选择,使得相容活动的集合最大?

贪心思路:按照活动结束时间进行顺序排序,然后从小到大依次选择相容活动。

\textit{思考:为什么不可以以开始时间最早、活动持续时间最短、与其他活动重叠最少来贪心?}

\paragraph{部分(分数)背包问题} 给定$n$种物品,其中第$i$种物品的价值为$v_i$,重量为$w_i$。现有一背包,可以装总质量不超过$C$的物品,其中每种物品可以取出部分装入背包。问如何选择装入背包的物品,使得在不超出背包容量的情况下,装入背包中的物品总价值最大?

贪心思路:计算出每种物品的单位价值,然后以从大到小的顺序依次装入背包,直到背包被装满。

\section{概率算法}

\subsection{Sherwood算法}

确定性算法通常假设算法的输入实例满足某一特定的概率分布。很多算法对不同输入实例运行时间差别很大(例如快速排序算法),可采用Sherwood概率算法消除时间复杂度与输入实例间的依赖关系:
\begin{itemize}
    \item 在确定性算法的某些步骤引入随机因素
    \item 对输入实例进行随机处理再进行确定性算法
\end{itemize}

\subsection{Las Vegas算法}
方式:重复随机决策直到获得了成功的结果再返回。

特点:要么返回正确的解,要么随机决策导致一个僵局。若陷入僵局使用同一实例运行同一算法有独立的机会求出解。成功的概率随执行时间的增大而增大。

对于实例$x$,设LV算法单次成功率为$p(x)$,成功时的期望时间为$s(x)$,失败时的期望时间为$e(x)$,则总期望时间为:
\begin{align*}
t(x)&=p(x)s(x)+(1-p(x))(e(x)+t(x)) \\
\Rightarrow t(x)&=s(x)+\frac{1-p(x)}{p(x)}e(x)
\end{align*}
如何最小化$t(x)$?

\subsection{Monte Carlo算法}

概念:
\begin{itemize}
    \item 设$p$是一个实数,且$1/2<p<1$,若一个Monte Carlo算法以不小于$p$的概率返回一个正确的解,则该MC算法称为$p$正确,算法的优势为$p-1/2$
    \item 若一个MC算法对同一实例绝不给出两个不同的正确解,则该算法称为相容的或一致的
\end{itemize}

一个MC算法是偏真的,指的是:算法可能把“真”错判为“假”,但绝不会把“假”错判为“真”。

重复调用$k$次一致、$p$正确、偏真的MC算法,可得到一个$(1-(1-p)^k)$正确的算法,即假设$k$次全是错误的结果,那么该算法的正确率为$(1-(1-p)^k)$。

如果要控制出错概率小于$\epsilon>0$,那么可以重复调用$k$次,其中
$$
(1-p)^k<\epsilon \Rightarrow k<\frac{\log\epsilon}{\log(1-p)}
$$
可取$\displaystyle k=\left\lceil\frac{\log\epsilon}{\log(1-p)}\right\rceil$

特点:偶尔会出错,但对于任何实例均能以高概率找到正确解。算法运行的次数越多,得到正确解的概率越高。

\subsection{例题}

\paragraph{数字积分}

对于定积分$\int_a^b f(x)dx$,求该积分的值可以在积分区间$[a, b]$上随机均匀地产生点,求这些点上地函数值的算术平均值,然后再乘以区间的宽度:
$$
\int_a^b f(x)\text{d}x=\frac{b-a}{n}\sum_{i=1}^n f(x_i), a\leq x_i\leq b
$$
将上式写为
$$
\int_a^b f(x)\text{d}x=\sum_{i=1}^n f(x_i)\frac{b-a}{n}, a\leq x_i\leq b
$$
等同于根据定积分的定义:
$$
\int_a^b f(x)\text{d}x=\lim_{\xi_i\rightarrow0}\sum_{i=1}^n f(\xi_i)\Delta x_i
$$
进行采样求解。

\begin{algorithm}[!h]
    \caption{概率算法进行数字积分计算定积分的值}
    \begin{algorithmic}
        \Require 函数$f$,采样量$n$,积分区间$a, b$
        \Ensure 函数$f$在区间$[a, b]$上的积分
        \State $\text{sum}\leftarrow 0$
        \For{$i\leftarrow 1$ to $n$}
            \State $x\leftarrow\text{uniform}(a, b)$
            \State $\text{sum}\leftarrow \text{sum} + f(x)$
        \EndFor
        \State\Return $(b-a)\text{sum} / n$
    \end{algorithmic}
\end{algorithm}

\paragraph{求集合的势(大小)}

设$X$为具有$n$个元素的集合。有放回地随机、均匀和独立地从$X$中选取元素,设$k$为出现第1次重复之前所选出的元素数目,则当$n$足够大时,$k$的期望趋近于$\beta\sqrt{n}$,其中$\beta=\sqrt{\pi/2}\approx1.253$\footnote{如何证明?}

于是可以使用以下公式设计估计$|X|$的概率算法:
$$
\beta\sqrt{n}=\sqrt{\frac{n\pi}{2}}=k\Rightarrow n=\frac{2k^2}{\pi}
$$

\paragraph{八皇后问题}

\begin{figure}[!h]
    \centering
    \includegraphics[width=\linewidth]{figures/queens.png}
    \caption{八皇后问题的LV算法伪代码}
    \label{fig:queens-lv}
\end{figure}

\paragraph{主元素问题} 设$A[1..n]$是含有$n$个元素的数组,若$A$中等于$x$的元素个数大于$n/2$,则称$x$是数组$A$的主元素。

MC算法:随机从$A$中选一个元素,然后计算这个元素的个数,并判断是否是主元素。是一个偏真$1/2$正确的算法。

\paragraph{矩阵乘法验证} 设$A, B, C$为$n\times n$矩阵,如何判定$AB=C$是否正确?

MC算法:构造一个随机$0/1$二值行向量,将判断$AB=C$改为判断$XAB=XC$。时间复杂度为$O(n^2)$。是一个偏假的正确概率为$1/2$的MC算法,因为当返回false时算法一定正确。

\section{分布式算法}

\subsection{模型}

考虑:
\begin{itemize}
    \item 异步消息传递模型:用于松散耦合机器及广域网
    \item 同步消息传递模型:一个理想的消息传递系统。该系统中,某些计时信息(如消息延迟上界)是已知的,系统的执行划分为轮执行,是异步系统的一种特例。
\end{itemize}

\subsection{基本算法}

\subsubsection{形式化模型(或许可以通过戴森球的一个产线来理解)}

\paragraph{拓扑} 无向图,其中节点表示处理机,边为双向信道。

\paragraph{算法} 由系统中每个处理器上的局部程序构成:一个系统或一个算法是由$n$个处理器$p_0, p_1, ..., p_{n-1}$构成,每个处理器$p_i$可以模型化为一个具有状态集$Q_i$的状态机(可能是无限的)。

\paragraph{局部程序} 负责执行局部计算,以及发送和接收消息。

\paragraph{状态(进程的局部状态)} 由$p_i$的变量,$p_i$的msgs构成。$p_i$的每个状态由$2r$个msg集构成:
\begin{itemize}
    \item $\text{outbuf}_i[l], 1\leq l\leq r$:$p_i$经第$l$条关联的信道发送给邻居,但尚未传到邻居的msg
    \item $\text{inbuf}_i[l], 1\leq l\leq r$:在$p_i$的第$l$条信道上已传递到$p_i$,但尚未经$p_i$内部计算步骤处理的msg。
\end{itemize}

\paragraph{初始状态} $Q_i$包含一个特殊的初始状态子集,其中每个$\text{inbuf}_i[l]$必须为空,但$\text{outbuf}_i[l]$未必为空。

\paragraph{转换函数(transition)} 处理器$p_i$的转换函数为:
\begin{itemize}
    \item 输入:$p_i$可访问的状态
    \item 输出:对每个信道$l$,至多产生一个msg输出
    \item 转换函数使$\text{inbuf}_i[l], 1\leq l\leq r$清空
\end{itemize}

\paragraph{配置} 配置是分布式系统在某点上整个算法的全局状态,表示为向量$(q_0, q_1, ..., q_{n-1})$,$q_i$是$p_i$的一个状态。一个配置里的outbuf变量的状态表示在通信信道上传输的信息,由del(传递)事件模拟传输。一个初始的配置为向量$(q_0, q_1, ..., q_{n-1})$,其中每个$q_i$是$p_i$的初始状态,即每个处理器处于初始状态。

\paragraph{事件} 系统里所发生的事情均称为事件,对于msg传递系统,有:
\begin{itemize}
    \item $\text{comp}(i)$:计算事件,代表处理器$p_i$的一个计算步骤。其中$p_i$的转换函数被用于当前可访问的状态
    \item $\text{del}(i, j, m)$:传递事件,表示msg m从$p_i$传送到$p_j$
\end{itemize}

\paragraph{执行} 系统在时间上的行为被称为一个执行。是一个由配置和事件交错构成的序列。该序列需主要满足以下两类条件:
\begin{itemize}
    \item Safety条件(安全性),表示某个性质在每次执行中每个可到达的配置里都必须成立。在序列的每个有限前缀里必须成立的条件
    \item Liveness条件(活跃性),表示某个性质在每次执行中的某些可达配置里必须成立。必须成立一定次数的条件(可能是无数次)
\end{itemize}
对特定系统,满足所有要求的安全性条件的序列称为一个\textbf{执行};若一个执行也满足所有要求的活跃性条件,则称为\textbf{容许(合法)执行}。

\subsubsection{同步系统和异步系统}



\printbibliography

\end{document}
